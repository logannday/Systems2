\documentclass{article}
\usepackage{graphicx}
\usepackage{subcaption}
\title{Simple Encoding and Decoding in C}
\author{Logan Day}
\date{\today}

\begin{document}
\maketitle
%...

\section{Reflection}
This project was a fun way to review file io in c. Judging by the sample output
for the test file given on canvas, it looks like my program properly encodes the
input data. Also, when I plug the encoded data back into my decoding program,
the original data is produced. I wrote a bash script in order to test the 
encoding and decoding efficiency of the program. The script appends 10,000
bytes of lorem ipsum text to an input file each time it runs. I also modified
my program to print the time it took to run the encoding/ decoding method, and the
file length in bytes to an output file, then plugged this file into google sheets
to generate charts. Both the data for encoding and decoding appears to be linear,
so I suspect my methods are O(n) with respect to the input size. 

\begin{figure}
  \includegraphics[width=\linewidth]{static/bash.png}
  \caption{Bash script used to run filesec on increasingly large input sizes. 10,000 bytes of
  randomly generated lorem text is added to the input file each time.}
  \label{fig:bash_script}
\end{figure}

\begin{figure}
  \includegraphics[width=\linewidth]{static/fig1.png}
  \caption{Running time for encoding (ms) charted against length of input over 100 trials.
  Each trial, the input file length was increased by 10,1000 bytes}
  \label{fig:encode}
\end{figure}

\begin{figure}
  \includegraphics[width=\linewidth]{static/fig2.png}
  \caption{Running time for decoding (ms) charted against length of input over 100 trials.
  Each trial, the input file length was increased by 10,1000 bytes}
  \label{fig:decode}
\end{figure}

%...
    
\end{document}
